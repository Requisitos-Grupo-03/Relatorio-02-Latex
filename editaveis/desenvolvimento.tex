
\chapter[Contexto do Negócio]{Contexto do Negócio}
\section{Introdução}
A organização FGRacing é uma equipe de competição da Universidade de Brasília -
Campus Gama. Atualmente, seu principal projeto é o desenvolvimento e construção
de um veículo Fórmula SAE elétrico. Este é um veículo especial para a competição
Fórmula SAE Brasil, e um protótipo dele pode ser observado na figura \ref{sae}.

A equipe é organizada de forma vertical, seguindo a seguinte ordem de hierarquia:

\begin{itemize}
  \item Capitão (cargo mais alto da empresa)
  \item Gerente de projetos
  \item Líderes (Membros os quais gerenciam os setores da empresa)
  \item Membros
\end{itemize}

Apesar da organização, a equipe enfrenta dificuldades na gerência dos projetos realizados. A integração entre os setores é falha,
 pois não existe uma comunicação eficiente entre eles. A falha na comunicação faz com que os membros sintam-se desmotivados,
  visto que a má integração das áreas causa a sensação de que o projeto está estagnado. Como consequência da desmotivação,
  os membros atrasam as tarefas, comprometendo o projeto, enquanto outros chegam a abandonar a equipe.

A empresa já tentou utilizar ferramentas de auxílio de gerência, porém a maioria não obteve sucesso. Entre elas pode-se
 citar: Trello, Slack, Whatsapp, Facebook e Google Drive. Dentre essas ferramentas, as únicas quem ainda se encontram
 em utilização são o Whatsapp e Google Drive. Quanto as que tiveram seu uso parado foram relatadas os seguintes problemas:

\begin{itemize}
  \item \textbf{Trello:} os líderes designavam atividades e tarefas para seus membros do setor , mas não davam feedback sobre suas execuções.
  \item \textbf{Slack:} os membros mostram muita resistência em migrar e se adaptar à nova plataforma.
  \item \textbf{Facebook:} utilizado apenas 2 vezes para fazer enquetes com a equipe.
\end{itemize}

Uma consequência da utilização de muitas ferramentas é a descentralização de informações. Isto torna árduo o controle de
informações, visto que não há padronização da divulgação destas, o que causa retrabalho na busca e compartilhamento da notícia
ou arquivo desejado.

O armazenamento de  documentos, .CADs e outros arquivos, é feito através do Google Drive. Entretanto, o compartilhamento de
 um arquivo nessa plataforma acaba sendo entre todos os membros da equipe, de forma que todos os membros têm acesso à
 documento que seriam “privados” de um setor específico. Isso acarreta numa falha de segurança: membros podem modificar, e
  até mesmo deletar, mesmo que não-intencionalmente, arquivos privados e cruciais para o projeto. Até mesmo usuários mal
  intencionados poderiam prejudicar a Equipe, ainda que não haja relatos desse tipo de experiência.

Outro ponto em que a equipe falha é no desenvolvimento de documentação interna, especialmente em atas de reuniões e documentos
 técnicos sobre o projeto. A consequência da ausência de documentação sobre as reuniões são: informações perdidas, tomadas
  de decisões erradas, o que de certo modo faz com as reuniões acabem não tendo um cunho oficial. Os documentos técnicos,
  por sua vez, são, geralmente, feitos tardiamente, apenas quando necessários. Isto prejudica a equipe principalmente em
  momentos em que é necessário apresentar o projeto para algum potencial patrocinador.

A gerência da FGRancing vem até então demonstrando dificuldades em administrar a equipe com as ferramentas já dispostas
 no mercado. Nesse cenário, se faz necessário a elaboração de um software que atenda às peculiaridades	administrativas
  da empresa.
  \pagebreak

\section{Problema}
Com a finalidade de traçar os sub-problemas e que levam a raiz do problema, a Equipe de Engenharia de Requisitos, em conjunto com o cliente, elaborou um Diagrama de Fishbone. Este diagrama, evoluído e priorizado com a professora e o cliente, permitiu evidenciar as raízes do problema maior da empresa: a falha na comunicação e transparência da equipe.

A representação gráfica deste diagrama é apresentada na Figura \ref{fishbone}.

\begin{figure}[!h]
        \centering
        \includegraphics[keepaspectratio=true,scale=0.9]{figuras/fishbone.eps}
        \caption{Diagrama de Fishbone\label{fishbone}}
\end{figure}


\section{Solução}
A solução proposta proposta foi o desenvolvimento de uma plataforma web capaz de integrar a gerência de atividades,
pessoas e arquivos. Esta solução está brevemente explicitada a seguir.

\subsection{Feed de Atividades}
A ferramenta irá dispor de um feed de atividades, onde todos os membros da equipe poderão visualizar o andamento
 dos projetos através do acompanhamento das ações realizadas. A ferramenta será capaz, ainda, de gerar
  relatórios automatizados sobre o status das atividades.

\subsection{Estrutura de Kanbans}
O sistema irá dispor de um esquema de kanbans estruturados conforme a hierarquia da empresa. O capitão poderá designar tarefas para os times,
bem como acompanhar seu desenvolvimento. Os líderes de setores, por sua vez, receberão as atividades a eles designadas,
 e distribuirão aos seus membros. Por fim, o time utiliza o kanban para organizar e informar a rastreabilidade das tarefas.

\subsection{Gerência de Arquivos e Documentos}
A entrega de uma atividade pode ser dependente da anexação de um arquivo. O sistema AutoTrack irá
gerenciar estes anexos, organizando-os conforme os setores da equipe, e suas áreas. Os arquivos terão um nível de privacidade.
 Assim, arquivos de setores específicos ficarão disponíveis para modificações apenas para membros deste setor, porém
  ficarão disponíveis para os membros de outros setores apenas para visualização.
